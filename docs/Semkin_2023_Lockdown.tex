\input{Preambula}
\usepackage[backend=biber]{biblatex}

\addbibresource{Semkin_2023_Lockdown.bib}

\begin{document}
	
	\section*{Introduction}
	
	В связи с известными событиями интерес к эпидемиологии и её методам сильно вырос. Для понимания динамики протекания короновирусной инфекции в 2020 году в разных местах Земли и на разных масштабах, а также для подготовки к будущим вспышкам заболеваемости, как никогда актуально построение адекватных математических моделей развития болезней в людских популяциях. Классические техники моделирования эпидемий опираются на параметризованные автономные системы дифференциальных уравнений, описывающие динамику изменения количества болеющих и здоровых людей. Эти модели дают хорошее понимание протекания болезни на больших масштабах (города, страны), но не способны описывать заболевание в небольших общественных структурах, например, промышленное предприятие, небольшую деревню или студенческое общежитие. В данной работе исследуется графовый подход к моделированию распространения инфекции, а именно вводится \textit{граф контактов}, по которому болезнь может "кочевать". В качестве представления болезни используется стандартная для эпидемиологии модель SIR/SEIR, в которой каждому человеку (вершине в графе) сопоставляется некоторое состояние (больной, здоровый и т.д.), после чего в дискретном времени происходят смены этих состояний с некоторыми вероятностями и система эволюционирует.
	
	Также исследуются различные эффекты от мер по борьбе с инфекцией, таких как тестирование, изоляция и, самое интересное, \textit{локдаун}. Именно он представляет основной интерес, так как его введение может привести к необычному последствию --- росту заболеваемости среди населения. Но обнаружить такое поведение в стандартных моделях не представляется возможным, поэтому цель данной работы --- найти условия возникновения такого эффекта в модели и продемонстрировать его на численных экспериментах. 
	
	Изучение эпидемий на больших популяциях позволяет моделировать этот процесс в среднем, и даже получать точные аналитические решения \cite{harko2014exact}. В зависимости от поставленной прикладной задачи возникает необходимость моделировать процесс эпидемии с разной степенью подробности. Так, например, простейшая модель SI \cite{allen1994some} рассматривает всего два состояния: больной и здоровый. В этой модели не рассматривается формирование иммунитета: здоровый всегда может заразиться при контакте с инфекцией. Существуют модели, рассматривающие дополнительно формирование иммунитета, инкубационный период, летальные исходы и многие другие возможные состояния. Одной из таких моделей является SEIR(S) \cite{capasso2008mathematical}. Моделирование в среднем не подходит для небольших или слишком разнородных популяций. Эту проблему позволяют решить модели распространения эпидемии на графах \cite{moreno2002epidemic}, \cite{pastor2015epidemic}. Распространение эпидемии на графе контактов можно рассматривать, например, при помощи цепи Маркова \cite{gomez2010discrete}. Однако моделирование распространения болезни на больших графах со сложной структурой имеет высокую алгоритмическую сложность. Наиболее распространенной является задача прогнозирования течения эпидемии \cite{leitch2019toward} и оценка индивидуальных рисков. Результаты изучения распространения эпидемии на графах	могут быть использованы не только для анализа заболеваний. Например, распространение слухов или автомобильного трафика можно описать схожим математическим аппаратом \cite{de2013anatomy}. Базой для данной статьи является \cite{base_article}, где, в частности, введена модель болезни на графе и где в её рамках исследован эффект локдауна.
	
	В работе ставится задача обобщить модель из \cite{base_article}, сформулировать новые условия возникновения роста заболеваемости при введении карантина и явно показать этот эффект в численном эксперименте. Т.о. появится возможность испытывать обновлённую модель в более широком спектре реальных ситуаций, а также пересмотреть локдаун как однозначно позитивную меру противодействия эпидемии.
	
	\printbibliography
	
\end{document}