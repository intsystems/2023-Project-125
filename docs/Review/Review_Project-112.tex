\input{MainPreambula}

\title{Рецензия на работу Дорина Д. <<Моделирование показания fMRI по видео, показанному человеку>>}
\author{Сёмкин К.}
\date{}

\begin{document}
	\maketitle
	
	\section*{Аннотация}
	
	Имеется ясное и сжатое описание научной работы, можно добавить пару слов о результатах проведённого анализа.
	
	\section*{Введение}
	
	Введение хорошо описывает тематику работы и конкретные сущности исследования, все ссылки релевантны и уместны. Поставленная задача обозначена. Основные замечания:
	
	\begin{itemize}
		\item можно добавить информацию о других подходах в обработке видеорядов
		\item информацию о результатах эксперимента можно вывести в заключение, описание выборки также можно исключить из введения, оно и так имеется в подходящей секции
	\end{itemize}

	\section*{Постановка задачи}
	
	\begin{itemize}
		\item какая именно есть связь между $\mathcal{S}_0$ и $\mathcal{S}$ ?
		\item индексы у элементов $\mathcal{S}$ и $\mathcal{P}$ можно сделать одинаковыми снизу
	\end{itemize}

	\section*{Описание модели}
	
	Обозначения читаемы и понятны, модель поставлена корректно. Процесс получения решения, а также его постобработка и метрики качества обозначены. Основные замечания:
	
	\begin{itemize}
		\item правильно ли понимаю, что число параметров $d \times X_s \times Y_s \times Z_s$ ? Насколько вычислительно сложна реальная модель? :) 
		\item кажется, что выбранный вид функции потерь (покомпонентный) имеет некоторое необозначенное предположение о связи параметров модели $w_{ijk}$. Потери без доп. предположений выглядели бы как сумма квадратов ошибок по всему тензору, т.е. по всем индексам $i, j, k$ ?
	\end{itemize}

	\section*{Эксперимент}
	
	\begin{itemize}
		\item можно поподробней описать демонстрацию работы алгоритма, т.е. для какого агента и в какой момент времени получен данный срез
	\end{itemize}

	\section*{Анализ ошибки}
	
	Всё отлично, графики читаемые и ясно отображают закономерности в зависимости ошибок от гиперпараметров. Все графики содержат достаточные описания. 
	
	\begin{itemize}
		\item интересно, что даже при задержке в 100 с MSE растёт всего лишь на $25e-6$ 
		\item про раздел 5.2: я не эксперт в статике, но на рис. 4 разве не продемонстрирована ЦПТ?)
		\item интересный результат о преемственности весов между разными агентами
		\item можно подробней обсудить неустойчивость модели к шуму, обосновать какой-нибудь математикой, что именно значит этот результат? 
	\end{itemize}

	\section*{Заключение}
	
	Можно написать о предполагаемых улучшениях и доработках модели. Также можно обсудить возрастание сложности модели при возрастании размерностей снимков фМРТ.
	
\end{document}